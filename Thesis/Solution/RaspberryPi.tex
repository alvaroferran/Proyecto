The Raspberry Pi carries out three main duties to ensure everything works correctly. These include creating a wireless connection, streaming images from the camera to the phone and transmitting the data received from the phone to the microcontroller.\\

 These are all placed into the \textit{/etc/rc.local} file so the system initializes them automatically each time the robot is turned on, with no need for human interaction.

\subsection{Wireless Communications}% WiFi Acces Point

The chosen means of communication between human and humanoid was wifi. This is so because it is a widely established technology, with great compatibility and in a great number of cases is already installed in the homes of users.\\

Three methods were considered: connection to an existing wifi network, creation of an Ad-Hoc connection and establishment of a wifi Access Point.

\subsubsection{Existing network:}

The most straightforward solution is to simply connect the robot to the user's existing wifi network. This enables the user to control it from anywhere in the world, expanding its uses. However, some configuration is required, namely selecting the desired network and introducing the password, which complicates the setup by having to add a keyboard and a display.\\

This method would thus be suitable for experienced users and developpers, but not necessarily the average seniors it is intended to help.

\subsubsection{Ad-Hoc connection:}



\subsubsection{Wifi Access Point:}




\subsection{MJPG Streamer}




\subsection{IP/UART Bridge} 
