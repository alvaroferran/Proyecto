\pagenumbering{gobble}% Remove page numbers (and reset to 1)
\vspace*{2cm}

\begin{center}
\color{part} \textsc{\huge \textbf{Resumen}}\\[1cm]
\end{center}


La edad media de los países desarrollados está aumentando, y tenderá a hacerlo aún más en el futuro. Con un número cada vez más importante de personas mayores con necesidad de asistencia, la demanda de ayuda está sobrepasando rápidamente la oferta disponible.\\

Para revertir la situación están siendo desarrollados robots capaces de asistir personas tanto emocional como físicamente. Estos robots satisfacerán las necesidades de los mayores, y siendo ayudantes artificiales se podrán construir los suficientes para satisfacer la demanda.\\

En este proyecto se desarrolla un prototipo de robot asistencial. A pesar de su tamaño relativamente pequeño, está programado teniendo en cuenta que el código se portará más adelante a un robot de tamaño humano, y por tanto es capaz de realizar lo mismo.\\

El Droide de Servicio Doméstico Personal (PD-SD por sus siglas en inglés) tiene un torso con brazos humanoide acoplado a una base con ruedas. Tiene dos brazos, cada uno con cinco grados de libertad que pueden ser utilizados para coger objetos o realizar acciones como cerrar puertas, mientras que la base móvil diferencial le permite maniobrar en espacios pequeños, ya que es capaz de rotar en el sitio.\\

El PD-SD se controla desde un teléfono con Android a través de una red wifi. La aplicación permite el control individual o por parejas simétricas de los motores de los brazos. Además permite controlar la base mediante un pad direccional, tiene un botón para abrir o cerrar las pinzas y otro para llevar al robot a su posición inicial. Finalmente, la parte superior de la pantalla está reservada para reproducir el vídeo recibido de la cámara de abordo.\\

En el propio robot un ordenador Raspberry Pi actúa de cerebro. Crea la red wifi a través de la cual recibe las órdenes y retransmite por ella el vídeo. Todo lo anterior está incluido en un script, con lo que se realiza automáticamente al encender el robot.\\

Cuando se establece una conexión entre el teléfono y el ordenador la pantalla LCD informará de ello mediante un mensaje, y hará lo propio cuando la conexión se cierre. Los datos recibidos se reenviarán mediante el puerto serie a un Arduino, que los parseará y controlará los actuadores pertinentes.